\documentclass{article}
\usepackage[utf8]{inputenc}
\usepackage[margin=1in]{geometry}
\usepackage{scrextend}
\usepackage[dvipsnames]{xcolor}
\usepackage{hyperref}
\usepackage{graphicx}
\usepackage{amsmath}
\hypersetup{
    colorlinks=true,
    linkcolor=blue,
    filecolor=magenta,      
    urlcolor=blue,
}
\setlength\parindent{0pt}
\usepackage{setspace}
\usepackage{color}

\begin{document}

\begin{center}
{\LARGE EAPS First Year Course 2019}\\
\vspace{10pt}
{\Large Writing and Resources at MIT}\\
\vspace{6pt}
\end{center}
Writing is an integral component of your PhD work. Below are some tips, instructions, and resources to help you write and cite everything from fellowship applications, to conference abstracts, to generals papers.\\

    \textbf{Recommended Downloads:}
    \begin{itemize}
        \item (Microsoft Office)
        \item \LaTeX + Overleaf (online)
        \item Mendeley
        \item Adobe Illustrator (CD in HQ)
        \item Google Scholar library add-on (online)
        \item \href{https://ist.mit.edu/cisco-anyconnect}{VPN}
    \end{itemize}

\section*{Typesetting}
First, \emph{where} will you write your words? In my experience, there are two primary options:
\textbf{Microsoft Word and \LaTeX}.
Your choice will affect how your writing is typeset and how your collaborators/editors/advisors provide feedback. MIT provides Microsoft Office (Word, Excel, Powerpoint, OneNote) to students free \href{https://ist.mit.edu/office/license}{at this link}. If you would like to download \LaTeX, there are a couple of options, but I recommend \href{https://pages.uoregon.edu/koch/texshop/}{TexShop}.\\

There are pros and cons to both word processing options: Word is probably more familiar to many of you, and arguably much easier to use. Many people prefer editing papers using Word, which allows for easy ``track changes'' and commenting. On the other hand, Latex is much better at typesetting equations\footnote{I have this very useful page bookmarked for easy access: \href{ https://en.wikibooks.org/wiki/LaTeX/Mathematics}{https://en.wikibooks.org/wiki/LaTeX/Mathematics}}, has a very straightforward way of referencing other (labeled) parts of the text, and allows you to customize your document to a level that I have never quite figured out in Word. If you are interested in a hands-on Latex tutorial, I am happy to sit down with you one-on-one or share some of my Latex files with you. (This document is available at github.com/juliawilcots/first-year-course if you'd like to look at the \texttt{.tex} file.)

\subsection*{Overleaf}
If you choose to write in Latex, MIT provides subscriptions to \href{https://www.overleaf.com}{Overleaf}, an online Latex editor. In addition to a more aesthetically pleasing interface than TexShop, Overleaf has many templates and compatibility with \href{https://www.sharelatex.com}{ShareLatex} and
\href{https://github.com}{GitHub}. By using Overleaf and GitHub together, you can easily edit the same document or project both on- and off-line. (If you're interested in learning how to do this, Google it or ask Julia.)

\section*{References}
Citation is an integral part of scientific writing that somehow always ends up taking longer than anticipated. Fortunately, there are a few tools that can help. Many researchers in EAPS use \href{https://libguides.mit.edu/cite-write/mendeley#s-lg-box-wrapper-18620527}{Mendeley}, a citation manager offered freely by the MIT Libraries. You can use Mendeley to store, organize, and export citations for PDFs. Noah Anderson (nanderso@) will give a Mendeley tutorial on 9/13/19 and you can email him with questions. If you are writing in Latex, you should export your bibliography as a BibTex \texttt{(.bib)} file. Talk to Julia if you have questions about adding your references in Latex -- it can be a pain. Another (less sleek) option is to manually add information about the papers you reference to a BibTex library. I use BibDesk (Figure \ref{fig:bib}), which comes in the TexShop distribution. 

\begin{figure}[h!]
    \centering
    \includegraphics[width=0.7\textwidth]{bibdesk.png}
    \caption{Example BibDesk entry. To manually add a reference you your library, click "New" (the plus on the window in the background) and then add information to at least all of the bolded fields in the window in the foreground.}
    \label{fig:bib}
\end{figure}

\subsection*{Finding papers}
Before you can cite a paper, you have to find it. I assume everyone is familiar with \href{https://scholar.google.com}{Google Scholar}, which is where I start. A lot of papers are (unfortunately) still stored behind paywalls. Fortunately for us, MIT subscribes to basically all journals and there are many easy ways of gaining access to papers off campus \footnote{On campus, you only need to be on the MIT, MIT Secure, or eduroam wifi networks to gain access.}:
\begin{enumerate}
    \item \textbf{Log into MIT wifi using a VPN}. This is an excellent option if you are travelling internationally! I highly recommend everyone download \href{https://ist.mit.edu/cisco-anyconnect}{Cisco AnyConnect}. You log in with your Kerberos ID, your password, and the word ``push'' as the second password. You'll then receive a DuoMobile push. 
    \item \textbf{Add MIT Libraries to your Google Scholar page}. See Figure \ref{fig:gs} for instructions on how to set this up.
    \item \textbf{Search the library catalog.} Sometimes Google Scholar is a pain, sometimes you need more than it can offer, sometimes you need...the library! We have two dedicated EAPS librarians and MIT has access to basically every book, thesis, map, or paper you could ever want. You can search for specific journals using \href{}{Vera}, order scans of print books/papers, or request a book through Inter-Library Loan. 
\end{enumerate}

\begin{figure}[h!]
    \centering
    \includegraphics[width=0.8\textwidth]{gs.png}
    \caption{To set up your Google Scholar account with MIT Libraries, go to Settings $>$ Library Links, then search for and add ``MIT''. Then, when you search for papers, click on ``Full Text - MIT Libraries'' and sign in with your usual MIT Duo login info.}
    \label{fig:gs}
\end{figure}

\section*{Figures}
Figures are another important aspect of your writing. We'll talk more about figure making on 11/1/19, but I have a few tips and suggestions in the meantime. I typically make figures using one of three programs: \href{https://matplotlib.org}{matplotlib} if I am making a plot or other computer-generated figure; \href{https://www.adobe.com/products/illustrator.html}{Adobe Illustrator} if I'm digitizing a sketch, annotating a photo, or beautifying a matplotlib plot; and Keynote (the Mac version of Powerpoint) when I want to do produce something that looks okay very quickly (e.g. Figure \ref{fig:gs}). If you would like the most updated version of Adobe Illustrator,  you can purchase a student license. However, \textbf{EAPS HQ has a copy of Illustrator (and other Adobe software) on CD}. You can borrow the disc (and a USB CD reader if you need it) from Maggie in HQ on the 9th floor and download Illustrator on your computer. You can also use Illustrator on any \href{https://ist.mit.edu/athena-clusters}{cluster computer} around MIT's campus. 

\section*{Improving your writing}
Since writing is such a critical component of what scientists do, a lot of people have thought hard about \emph{how} to actually go about writing scientific papers. The book \emph{Eloquent Science: A practical guide to becoming a better writer, speaker, and atmospheric scientist} by David M. Schultz (2009) is a nice resource if you are looking for suggestions on how to improve your writing. MIT also has a \href{https://cmsw.mit.edu/writing-and-communication-center/}{writing center}, where you can schedule 1-on-1 appointments to talk over your written work.

\section*{Fellowships}
As a graduate student in EAPS, you are guaranteed five years of funding from your advisor. However, it is generally considered a good idea to apply for fellowships (but this is something you should talk to your advisor about!!). There are lots and lots of fellowhsip options, both internal to MIT and external. Some of the most popular include:
\begin{itemize}
    \item National Science Foundation Graduate Research Fellowship Program (NSF GRFP) (Geosciences due date is mid to late October; you may apply \textbf{once} as a graduate student)
    \item National Defense Science and Engineering Graduate Fellowship Program (\href{https://www.ndsegfellowships.org}{NDSEG}) (due in December/January)
    \item DOE Computational Scie
\end{itemize}

\end{document}

